%%% generic article type (pdf)latex file
%%% use together with Makefile

\documentclass[letterpaper]{scrartcl}
\usepackage{graphicx}
\usepackage{amsmath,amsfonts,amsthm}
\usepackage{eufrak}
\usepackage{mathabx}
\usepackage{url}
\usepackage[colorlinks]{hyperref}
\usepackage{enumitem}
\usepackage{booktabs}
\usepackage{listings}

\lstset{language=sh}

%\usepackage{wrapfig}
\usepackage{subfig}
\usepackage[format=plain,labelsep=period,font=small,labelfont=bf]{caption}

%------------------------------------------------------------
% assignment
%
\newcommand{\anumber}{1}
%
%------------------------------------------------------------

\newcommand{\BONUS}{\textsc{Bonus: }}
\newcommand{\bonus}[1]{\textbf{[bonus +#1*]}}
\newcommand{\points}[1]{\textbf{[#1 points]}}
\newenvironment{enuma}{\begin{enumerate}[label=(\alph*)]}{\end{enumerate}}
\newenvironment{enumi}{\begin{enumerate}[label=(\roman*)]}{\end{enumerate}}
\newenvironment{solution}{\par\noindent\P{} }{\ \qedsymbol}

\renewcommand{\vec}[1]{\ensuremath{\mathbf{#1}}}
\newcommand{\pd}[3][]{\left(\frac{\partial #2}{\partial #3}\right)_{#1}}

\newcommand{\fnhref}[2]{\href{#1}{#2}\footnote{\url{#1}}}


\begin{document}
%\maketitle

\setcounter{section}{\anumber}
\addtocounter{section}{-1}
\section{ --- PHY 494: Homework assignment (15 points total)}

\noindent Due Thursday, Jan 17, 2019, 1:30pm.

\noindent Submit a PDF through Canvas (name it
\texttt{\emph{lastname}\_\emph{firstname}\_hw\anumber.pdf}).
Homeworks must be legible or may otherwise be returned ungraded with 0
points.

This assignment contains \textbf{bonus problems}. A bonus problem is
optional. If you do it you get additional points that count towards
this homework's total, although you can't get more than the maximum
number of points. If you don't do it you can still get full
points. Bonus problems and bonus points are indicated with an asterisk
``*''.

Note: In general, for full credit you need to (1) show the commands
that you used and (2) answer the question. Sometimes you should also
copy and paste output.

\subsection{Commands and paths (8 points)}
(The following questions do not require you to show code unless
explicitly stated.)
\begin{enuma}
\item Briefly describe the function of the \texttt{cd} and the
  \texttt{pwd} command? \points{2}
\item Show commands for two different ways to change to your home directory,
  assuming you are currently in the root directory. \points{1.5}

  \BONUS Show a third possibility. \bonus{0.5}
\item Given the path \texttt{/home/dvader/Documents/../data/bases}:
  \begin{enumi}
    \item Is this an absolute or relative path? \points{0.5}
    \item If you are located in the home directory of user dvader
      (\texttt{/home/dvader}) then what is the shortest path to
      \texttt{bases}, i.e., show the command to change to the
      \texttt{bases} directory. \points{1}
  \end{enumi}
\item If you were in a directory \texttt{/home/dvader/data} and you
  executed the command \texttt{cd ./.././.././.}, what would be the
  output of running the \texttt{pwd} command afterwards? \points{1}
\item Describe two ways by which you could learn more about the
  function of a Unix command \texttt{frbzz} that you don't know
  anything about. \points{2}
\item \BONUS (Skim)read Neal Stephenson's \emph{In the Beginning was
  the command line} from 1999
  (\fnhref{https://becksteinlab.physics.asu.edu/file_download/7/NealStephenson_Commandline.pdf?mimetype=pdf}{PDF})\footnote{originally
  available from
  \url{http://www.cryptonomicon.com/beginning.html}}. What are the
  advantages and disadvantages of using the command line instead of a
  graphical user interface? \bonus{4}
\end{enuma}

\subsection{Copy, rename, delete (4 points)}

Work through the
\fnhref{https://asu-compmethodsphysics-phy494.github.io/ASU-PHY494/2019/01/10/01_Unix_Shell/\#copy-rename-delete}{Copy,
  rename, delete: Activity} (note that this exercise builds on
previous parts of
\fnhref{https://asu-compmethodsphysics-phy494.github.io/ASU-PHY494/2019/01/10/01_Unix_Shell/}{01
  The Unix Shell}, which you should have also done). After you
completed the activity (points 1 -- 11) you should end up with a
specific directory structure under
\texttt{$\sim$/PHY494/01\_shell}. Show the output of the commands
\begin{lstlisting}
  cd ~
  ls -R PHY494/01_shell
\end{lstlisting}
which will be compared against the expected directory structure and
content. \points{4}

\subsection{Danger Zone (3 points)}

\begin{center}
  \fbox{\parbox{\textwidth}{\textbf{DO NOT TRY THE FOLLOWING COMMAND JUST
      TO FIND OUT WHAT IT DOES. You have been warned!}}}
\end{center}
\textbf{Describe} what the command \texttt{rm -rf /} might do. Should
you \emph{ever} use it?

\subsection{BONUS: Pipes and Filters (+5* points)}
\label{sec:pipesandfilters}

Work through the activities in the section
\fnhref{https://asu-compmethodsphysics-phy494.github.io/ASU-PHY494/2019/01/10/01_Unix_Shell/\#pipes-and-filters}{Pipes
  and Filters}. Answer the following questions and show the commands
that you used to arrive at the answer.

\begin{enuma}
\item How many lines does the file \texttt{planets\_2.dat} contain? \bonus{1}
\item What are the three biggest planets (by diameter) in the file
  \texttt{planets.dat}? \bonus{1}
\item Which planets contain \emph{ice} terrain? \bonus{1}
\item \label{li:characters} What is the most frequent and the least
  frequent first letter amongst \emph{all} the planets? \bonus{2}
\end{enuma}



\end{document}

%%% Local Variables: 
%%% mode: latex
%%% TeX-master: t
%%% End: 
