
% Default to the notebook output style

    


% Inherit from the specified cell style.




    
\documentclass[11pt]{article}

    
    
    \usepackage[T1]{fontenc}
    % Nicer default font (+ math font) than Computer Modern for most use cases
    \usepackage{mathpazo}

    % Basic figure setup, for now with no caption control since it's done
    % automatically by Pandoc (which extracts ![](path) syntax from Markdown).
    \usepackage{graphicx}
    % We will generate all images so they have a width \maxwidth. This means
    % that they will get their normal width if they fit onto the page, but
    % are scaled down if they would overflow the margins.
    \makeatletter
    \def\maxwidth{\ifdim\Gin@nat@width>\linewidth\linewidth
    \else\Gin@nat@width\fi}
    \makeatother
    \let\Oldincludegraphics\includegraphics
    % Set max figure width to be 80% of text width, for now hardcoded.
    \renewcommand{\includegraphics}[1]{\Oldincludegraphics[width=.8\maxwidth]{#1}}
    % Ensure that by default, figures have no caption (until we provide a
    % proper Figure object with a Caption API and a way to capture that
    % in the conversion process - todo).
    \usepackage{caption}
    \DeclareCaptionLabelFormat{nolabel}{}
    \captionsetup{labelformat=nolabel}

    \usepackage{adjustbox} % Used to constrain images to a maximum size 
    \usepackage{xcolor} % Allow colors to be defined
    \usepackage{enumerate} % Needed for markdown enumerations to work
    \usepackage{geometry} % Used to adjust the document margins
    \usepackage{amsmath} % Equations
    \usepackage{amssymb} % Equations
    \usepackage{textcomp} % defines textquotesingle
    % Hack from http://tex.stackexchange.com/a/47451/13684:
    \AtBeginDocument{%
        \def\PYZsq{\textquotesingle}% Upright quotes in Pygmentized code
    }
    \usepackage{upquote} % Upright quotes for verbatim code
    \usepackage{eurosym} % defines \euro
    \usepackage[mathletters]{ucs} % Extended unicode (utf-8) support
    \usepackage[utf8x]{inputenc} % Allow utf-8 characters in the tex document
    \usepackage{fancyvrb} % verbatim replacement that allows latex
    \usepackage{grffile} % extends the file name processing of package graphics 
                         % to support a larger range 
    % The hyperref package gives us a pdf with properly built
    % internal navigation ('pdf bookmarks' for the table of contents,
    % internal cross-reference links, web links for URLs, etc.)
    \usepackage{hyperref}
    \usepackage{longtable} % longtable support required by pandoc >1.10
    \usepackage{booktabs}  % table support for pandoc > 1.12.2
    \usepackage[inline]{enumitem} % IRkernel/repr support (it uses the enumerate* environment)
    \usepackage[normalem]{ulem} % ulem is needed to support strikethroughs (\sout)
                                % normalem makes italics be italics, not underlines
    \usepackage{mathrsfs}
    

    
    
    % Colors for the hyperref package
    \definecolor{urlcolor}{rgb}{0,.145,.698}
    \definecolor{linkcolor}{rgb}{.71,0.21,0.01}
    \definecolor{citecolor}{rgb}{.12,.54,.11}

    % ANSI colors
    \definecolor{ansi-black}{HTML}{3E424D}
    \definecolor{ansi-black-intense}{HTML}{282C36}
    \definecolor{ansi-red}{HTML}{E75C58}
    \definecolor{ansi-red-intense}{HTML}{B22B31}
    \definecolor{ansi-green}{HTML}{00A250}
    \definecolor{ansi-green-intense}{HTML}{007427}
    \definecolor{ansi-yellow}{HTML}{DDB62B}
    \definecolor{ansi-yellow-intense}{HTML}{B27D12}
    \definecolor{ansi-blue}{HTML}{208FFB}
    \definecolor{ansi-blue-intense}{HTML}{0065CA}
    \definecolor{ansi-magenta}{HTML}{D160C4}
    \definecolor{ansi-magenta-intense}{HTML}{A03196}
    \definecolor{ansi-cyan}{HTML}{60C6C8}
    \definecolor{ansi-cyan-intense}{HTML}{258F8F}
    \definecolor{ansi-white}{HTML}{C5C1B4}
    \definecolor{ansi-white-intense}{HTML}{A1A6B2}
    \definecolor{ansi-default-inverse-fg}{HTML}{FFFFFF}
    \definecolor{ansi-default-inverse-bg}{HTML}{000000}

    % commands and environments needed by pandoc snippets
    % extracted from the output of `pandoc -s`
    \providecommand{\tightlist}{%
      \setlength{\itemsep}{0pt}\setlength{\parskip}{0pt}}
    \DefineVerbatimEnvironment{Highlighting}{Verbatim}{commandchars=\\\{\}}
    % Add ',fontsize=\small' for more characters per line
    \newenvironment{Shaded}{}{}
    \newcommand{\KeywordTok}[1]{\textcolor[rgb]{0.00,0.44,0.13}{\textbf{{#1}}}}
    \newcommand{\DataTypeTok}[1]{\textcolor[rgb]{0.56,0.13,0.00}{{#1}}}
    \newcommand{\DecValTok}[1]{\textcolor[rgb]{0.25,0.63,0.44}{{#1}}}
    \newcommand{\BaseNTok}[1]{\textcolor[rgb]{0.25,0.63,0.44}{{#1}}}
    \newcommand{\FloatTok}[1]{\textcolor[rgb]{0.25,0.63,0.44}{{#1}}}
    \newcommand{\CharTok}[1]{\textcolor[rgb]{0.25,0.44,0.63}{{#1}}}
    \newcommand{\StringTok}[1]{\textcolor[rgb]{0.25,0.44,0.63}{{#1}}}
    \newcommand{\CommentTok}[1]{\textcolor[rgb]{0.38,0.63,0.69}{\textit{{#1}}}}
    \newcommand{\OtherTok}[1]{\textcolor[rgb]{0.00,0.44,0.13}{{#1}}}
    \newcommand{\AlertTok}[1]{\textcolor[rgb]{1.00,0.00,0.00}{\textbf{{#1}}}}
    \newcommand{\FunctionTok}[1]{\textcolor[rgb]{0.02,0.16,0.49}{{#1}}}
    \newcommand{\RegionMarkerTok}[1]{{#1}}
    \newcommand{\ErrorTok}[1]{\textcolor[rgb]{1.00,0.00,0.00}{\textbf{{#1}}}}
    \newcommand{\NormalTok}[1]{{#1}}
    
    % Additional commands for more recent versions of Pandoc
    \newcommand{\ConstantTok}[1]{\textcolor[rgb]{0.53,0.00,0.00}{{#1}}}
    \newcommand{\SpecialCharTok}[1]{\textcolor[rgb]{0.25,0.44,0.63}{{#1}}}
    \newcommand{\VerbatimStringTok}[1]{\textcolor[rgb]{0.25,0.44,0.63}{{#1}}}
    \newcommand{\SpecialStringTok}[1]{\textcolor[rgb]{0.73,0.40,0.53}{{#1}}}
    \newcommand{\ImportTok}[1]{{#1}}
    \newcommand{\DocumentationTok}[1]{\textcolor[rgb]{0.73,0.13,0.13}{\textit{{#1}}}}
    \newcommand{\AnnotationTok}[1]{\textcolor[rgb]{0.38,0.63,0.69}{\textbf{\textit{{#1}}}}}
    \newcommand{\CommentVarTok}[1]{\textcolor[rgb]{0.38,0.63,0.69}{\textbf{\textit{{#1}}}}}
    \newcommand{\VariableTok}[1]{\textcolor[rgb]{0.10,0.09,0.49}{{#1}}}
    \newcommand{\ControlFlowTok}[1]{\textcolor[rgb]{0.00,0.44,0.13}{\textbf{{#1}}}}
    \newcommand{\OperatorTok}[1]{\textcolor[rgb]{0.40,0.40,0.40}{{#1}}}
    \newcommand{\BuiltInTok}[1]{{#1}}
    \newcommand{\ExtensionTok}[1]{{#1}}
    \newcommand{\PreprocessorTok}[1]{\textcolor[rgb]{0.74,0.48,0.00}{{#1}}}
    \newcommand{\AttributeTok}[1]{\textcolor[rgb]{0.49,0.56,0.16}{{#1}}}
    \newcommand{\InformationTok}[1]{\textcolor[rgb]{0.38,0.63,0.69}{\textbf{\textit{{#1}}}}}
    \newcommand{\WarningTok}[1]{\textcolor[rgb]{0.38,0.63,0.69}{\textbf{\textit{{#1}}}}}
    
    
    % Define a nice break command that doesn't care if a line doesn't already
    % exist.
    \def\br{\hspace*{\fill} \\* }
    % Math Jax compatibility definitions
    \def\gt{>}
    \def\lt{<}
    \let\Oldtex\TeX
    \let\Oldlatex\LaTeX
    \renewcommand{\TeX}{\textrm{\Oldtex}}
    \renewcommand{\LaTeX}{\textrm{\Oldlatex}}
    % Document parameters
    % Document title
    \title{project\_1}
    
    
    
    
    

    % Pygments definitions
    
\makeatletter
\def\PY@reset{\let\PY@it=\relax \let\PY@bf=\relax%
    \let\PY@ul=\relax \let\PY@tc=\relax%
    \let\PY@bc=\relax \let\PY@ff=\relax}
\def\PY@tok#1{\csname PY@tok@#1\endcsname}
\def\PY@toks#1+{\ifx\relax#1\empty\else%
    \PY@tok{#1}\expandafter\PY@toks\fi}
\def\PY@do#1{\PY@bc{\PY@tc{\PY@ul{%
    \PY@it{\PY@bf{\PY@ff{#1}}}}}}}
\def\PY#1#2{\PY@reset\PY@toks#1+\relax+\PY@do{#2}}

\expandafter\def\csname PY@tok@w\endcsname{\def\PY@tc##1{\textcolor[rgb]{0.73,0.73,0.73}{##1}}}
\expandafter\def\csname PY@tok@c\endcsname{\let\PY@it=\textit\def\PY@tc##1{\textcolor[rgb]{0.25,0.50,0.50}{##1}}}
\expandafter\def\csname PY@tok@cp\endcsname{\def\PY@tc##1{\textcolor[rgb]{0.74,0.48,0.00}{##1}}}
\expandafter\def\csname PY@tok@k\endcsname{\let\PY@bf=\textbf\def\PY@tc##1{\textcolor[rgb]{0.00,0.50,0.00}{##1}}}
\expandafter\def\csname PY@tok@kp\endcsname{\def\PY@tc##1{\textcolor[rgb]{0.00,0.50,0.00}{##1}}}
\expandafter\def\csname PY@tok@kt\endcsname{\def\PY@tc##1{\textcolor[rgb]{0.69,0.00,0.25}{##1}}}
\expandafter\def\csname PY@tok@o\endcsname{\def\PY@tc##1{\textcolor[rgb]{0.40,0.40,0.40}{##1}}}
\expandafter\def\csname PY@tok@ow\endcsname{\let\PY@bf=\textbf\def\PY@tc##1{\textcolor[rgb]{0.67,0.13,1.00}{##1}}}
\expandafter\def\csname PY@tok@nb\endcsname{\def\PY@tc##1{\textcolor[rgb]{0.00,0.50,0.00}{##1}}}
\expandafter\def\csname PY@tok@nf\endcsname{\def\PY@tc##1{\textcolor[rgb]{0.00,0.00,1.00}{##1}}}
\expandafter\def\csname PY@tok@nc\endcsname{\let\PY@bf=\textbf\def\PY@tc##1{\textcolor[rgb]{0.00,0.00,1.00}{##1}}}
\expandafter\def\csname PY@tok@nn\endcsname{\let\PY@bf=\textbf\def\PY@tc##1{\textcolor[rgb]{0.00,0.00,1.00}{##1}}}
\expandafter\def\csname PY@tok@ne\endcsname{\let\PY@bf=\textbf\def\PY@tc##1{\textcolor[rgb]{0.82,0.25,0.23}{##1}}}
\expandafter\def\csname PY@tok@nv\endcsname{\def\PY@tc##1{\textcolor[rgb]{0.10,0.09,0.49}{##1}}}
\expandafter\def\csname PY@tok@no\endcsname{\def\PY@tc##1{\textcolor[rgb]{0.53,0.00,0.00}{##1}}}
\expandafter\def\csname PY@tok@nl\endcsname{\def\PY@tc##1{\textcolor[rgb]{0.63,0.63,0.00}{##1}}}
\expandafter\def\csname PY@tok@ni\endcsname{\let\PY@bf=\textbf\def\PY@tc##1{\textcolor[rgb]{0.60,0.60,0.60}{##1}}}
\expandafter\def\csname PY@tok@na\endcsname{\def\PY@tc##1{\textcolor[rgb]{0.49,0.56,0.16}{##1}}}
\expandafter\def\csname PY@tok@nt\endcsname{\let\PY@bf=\textbf\def\PY@tc##1{\textcolor[rgb]{0.00,0.50,0.00}{##1}}}
\expandafter\def\csname PY@tok@nd\endcsname{\def\PY@tc##1{\textcolor[rgb]{0.67,0.13,1.00}{##1}}}
\expandafter\def\csname PY@tok@s\endcsname{\def\PY@tc##1{\textcolor[rgb]{0.73,0.13,0.13}{##1}}}
\expandafter\def\csname PY@tok@sd\endcsname{\let\PY@it=\textit\def\PY@tc##1{\textcolor[rgb]{0.73,0.13,0.13}{##1}}}
\expandafter\def\csname PY@tok@si\endcsname{\let\PY@bf=\textbf\def\PY@tc##1{\textcolor[rgb]{0.73,0.40,0.53}{##1}}}
\expandafter\def\csname PY@tok@se\endcsname{\let\PY@bf=\textbf\def\PY@tc##1{\textcolor[rgb]{0.73,0.40,0.13}{##1}}}
\expandafter\def\csname PY@tok@sr\endcsname{\def\PY@tc##1{\textcolor[rgb]{0.73,0.40,0.53}{##1}}}
\expandafter\def\csname PY@tok@ss\endcsname{\def\PY@tc##1{\textcolor[rgb]{0.10,0.09,0.49}{##1}}}
\expandafter\def\csname PY@tok@sx\endcsname{\def\PY@tc##1{\textcolor[rgb]{0.00,0.50,0.00}{##1}}}
\expandafter\def\csname PY@tok@m\endcsname{\def\PY@tc##1{\textcolor[rgb]{0.40,0.40,0.40}{##1}}}
\expandafter\def\csname PY@tok@gh\endcsname{\let\PY@bf=\textbf\def\PY@tc##1{\textcolor[rgb]{0.00,0.00,0.50}{##1}}}
\expandafter\def\csname PY@tok@gu\endcsname{\let\PY@bf=\textbf\def\PY@tc##1{\textcolor[rgb]{0.50,0.00,0.50}{##1}}}
\expandafter\def\csname PY@tok@gd\endcsname{\def\PY@tc##1{\textcolor[rgb]{0.63,0.00,0.00}{##1}}}
\expandafter\def\csname PY@tok@gi\endcsname{\def\PY@tc##1{\textcolor[rgb]{0.00,0.63,0.00}{##1}}}
\expandafter\def\csname PY@tok@gr\endcsname{\def\PY@tc##1{\textcolor[rgb]{1.00,0.00,0.00}{##1}}}
\expandafter\def\csname PY@tok@ge\endcsname{\let\PY@it=\textit}
\expandafter\def\csname PY@tok@gs\endcsname{\let\PY@bf=\textbf}
\expandafter\def\csname PY@tok@gp\endcsname{\let\PY@bf=\textbf\def\PY@tc##1{\textcolor[rgb]{0.00,0.00,0.50}{##1}}}
\expandafter\def\csname PY@tok@go\endcsname{\def\PY@tc##1{\textcolor[rgb]{0.53,0.53,0.53}{##1}}}
\expandafter\def\csname PY@tok@gt\endcsname{\def\PY@tc##1{\textcolor[rgb]{0.00,0.27,0.87}{##1}}}
\expandafter\def\csname PY@tok@err\endcsname{\def\PY@bc##1{\setlength{\fboxsep}{0pt}\fcolorbox[rgb]{1.00,0.00,0.00}{1,1,1}{\strut ##1}}}
\expandafter\def\csname PY@tok@kc\endcsname{\let\PY@bf=\textbf\def\PY@tc##1{\textcolor[rgb]{0.00,0.50,0.00}{##1}}}
\expandafter\def\csname PY@tok@kd\endcsname{\let\PY@bf=\textbf\def\PY@tc##1{\textcolor[rgb]{0.00,0.50,0.00}{##1}}}
\expandafter\def\csname PY@tok@kn\endcsname{\let\PY@bf=\textbf\def\PY@tc##1{\textcolor[rgb]{0.00,0.50,0.00}{##1}}}
\expandafter\def\csname PY@tok@kr\endcsname{\let\PY@bf=\textbf\def\PY@tc##1{\textcolor[rgb]{0.00,0.50,0.00}{##1}}}
\expandafter\def\csname PY@tok@bp\endcsname{\def\PY@tc##1{\textcolor[rgb]{0.00,0.50,0.00}{##1}}}
\expandafter\def\csname PY@tok@fm\endcsname{\def\PY@tc##1{\textcolor[rgb]{0.00,0.00,1.00}{##1}}}
\expandafter\def\csname PY@tok@vc\endcsname{\def\PY@tc##1{\textcolor[rgb]{0.10,0.09,0.49}{##1}}}
\expandafter\def\csname PY@tok@vg\endcsname{\def\PY@tc##1{\textcolor[rgb]{0.10,0.09,0.49}{##1}}}
\expandafter\def\csname PY@tok@vi\endcsname{\def\PY@tc##1{\textcolor[rgb]{0.10,0.09,0.49}{##1}}}
\expandafter\def\csname PY@tok@vm\endcsname{\def\PY@tc##1{\textcolor[rgb]{0.10,0.09,0.49}{##1}}}
\expandafter\def\csname PY@tok@sa\endcsname{\def\PY@tc##1{\textcolor[rgb]{0.73,0.13,0.13}{##1}}}
\expandafter\def\csname PY@tok@sb\endcsname{\def\PY@tc##1{\textcolor[rgb]{0.73,0.13,0.13}{##1}}}
\expandafter\def\csname PY@tok@sc\endcsname{\def\PY@tc##1{\textcolor[rgb]{0.73,0.13,0.13}{##1}}}
\expandafter\def\csname PY@tok@dl\endcsname{\def\PY@tc##1{\textcolor[rgb]{0.73,0.13,0.13}{##1}}}
\expandafter\def\csname PY@tok@s2\endcsname{\def\PY@tc##1{\textcolor[rgb]{0.73,0.13,0.13}{##1}}}
\expandafter\def\csname PY@tok@sh\endcsname{\def\PY@tc##1{\textcolor[rgb]{0.73,0.13,0.13}{##1}}}
\expandafter\def\csname PY@tok@s1\endcsname{\def\PY@tc##1{\textcolor[rgb]{0.73,0.13,0.13}{##1}}}
\expandafter\def\csname PY@tok@mb\endcsname{\def\PY@tc##1{\textcolor[rgb]{0.40,0.40,0.40}{##1}}}
\expandafter\def\csname PY@tok@mf\endcsname{\def\PY@tc##1{\textcolor[rgb]{0.40,0.40,0.40}{##1}}}
\expandafter\def\csname PY@tok@mh\endcsname{\def\PY@tc##1{\textcolor[rgb]{0.40,0.40,0.40}{##1}}}
\expandafter\def\csname PY@tok@mi\endcsname{\def\PY@tc##1{\textcolor[rgb]{0.40,0.40,0.40}{##1}}}
\expandafter\def\csname PY@tok@il\endcsname{\def\PY@tc##1{\textcolor[rgb]{0.40,0.40,0.40}{##1}}}
\expandafter\def\csname PY@tok@mo\endcsname{\def\PY@tc##1{\textcolor[rgb]{0.40,0.40,0.40}{##1}}}
\expandafter\def\csname PY@tok@ch\endcsname{\let\PY@it=\textit\def\PY@tc##1{\textcolor[rgb]{0.25,0.50,0.50}{##1}}}
\expandafter\def\csname PY@tok@cm\endcsname{\let\PY@it=\textit\def\PY@tc##1{\textcolor[rgb]{0.25,0.50,0.50}{##1}}}
\expandafter\def\csname PY@tok@cpf\endcsname{\let\PY@it=\textit\def\PY@tc##1{\textcolor[rgb]{0.25,0.50,0.50}{##1}}}
\expandafter\def\csname PY@tok@c1\endcsname{\let\PY@it=\textit\def\PY@tc##1{\textcolor[rgb]{0.25,0.50,0.50}{##1}}}
\expandafter\def\csname PY@tok@cs\endcsname{\let\PY@it=\textit\def\PY@tc##1{\textcolor[rgb]{0.25,0.50,0.50}{##1}}}

\def\PYZbs{\char`\\}
\def\PYZus{\char`\_}
\def\PYZob{\char`\{}
\def\PYZcb{\char`\}}
\def\PYZca{\char`\^}
\def\PYZam{\char`\&}
\def\PYZlt{\char`\<}
\def\PYZgt{\char`\>}
\def\PYZsh{\char`\#}
\def\PYZpc{\char`\%}
\def\PYZdl{\char`\$}
\def\PYZhy{\char`\-}
\def\PYZsq{\char`\'}
\def\PYZdq{\char`\"}
\def\PYZti{\char`\~}
% for compatibility with earlier versions
\def\PYZat{@}
\def\PYZlb{[}
\def\PYZrb{]}
\makeatother


    % Exact colors from NB
    \definecolor{incolor}{rgb}{0.0, 0.0, 0.5}
    \definecolor{outcolor}{rgb}{0.545, 0.0, 0.0}



    
    % Prevent overflowing lines due to hard-to-break entities
    \sloppy 
    % Setup hyperref package
    \hypersetup{
      breaklinks=true,  % so long urls are correctly broken across lines
      colorlinks=true,
      urlcolor=urlcolor,
      linkcolor=linkcolor,
      citecolor=citecolor,
      }
    % Slightly bigger margins than the latex defaults
    
    \geometry{verbose,tmargin=1in,bmargin=1in,lmargin=1in,rmargin=1in}
    
    

    \begin{document}
    
    
    \maketitle
    
    

    
    \section{Project 1}\label{project-1}

    \subsection{0. Understanding the
Problem}\label{understanding-the-problem}

    We are given a number of constants that help identify the best way to
solve the kinematics problem. We assign all given constants to a
variable and import all modules that will be necessary.

    \begin{Verbatim}[commandchars=\\\{\}]
{\color{incolor}In [{\color{incolor}15}]:} \PY{k+kn}{import} \PY{n+nn}{matplotlib}\PY{n+nn}{.}\PY{n+nn}{pyplot} \PY{k}{as} \PY{n+nn}{plt}
         \PY{k+kn}{import} \PY{n+nn}{numpy} \PY{k}{as} \PY{n+nn}{np}
         
         \PY{n}{a} \PY{o}{=} \PY{o}{\PYZhy{}}\PY{l+m+mi}{3}  \PY{c+c1}{\PYZsh{} vehicle decceleration (m/s\PYZca{}2)}
         \PY{n}{v0} \PY{o}{=} \PY{l+m+mf}{15.2778}  \PY{c+c1}{\PYZsh{} vehicle velocity (m/s)}
         \PY{n}{tau} \PY{o}{=} \PY{l+m+mi}{3}  \PY{c+c1}{\PYZsh{} green to red (s)}
         \PY{n}{d} \PY{o}{=} \PY{l+m+mf}{0.8}  \PY{c+c1}{\PYZsh{} reaction time (s)}
         \PY{n}{w} \PY{o}{=} \PY{l+m+mi}{30}  \PY{c+c1}{\PYZsh{} intersection width (m)}
\end{Verbatim}

    \subsection{1. Equations of motion}\label{equations-of-motion}

    We begin by creating the equations of motion for the vehicle. We must
split the vehicles trajectory into three parts. The first part is when
the vehicle moves at constant velocity before the traffic signal changes
color:

\(x(t) = -x_0 + v_0t\) , where \(v_0 = 15.3\) \(m/s\)

The next part is for the vehicle's trajectory if the driver decides to
decelerate:

\(x(t) = -x_0 + v_0t - \frac{a}{2}t^2\) \& \(v(t) = v_0 - at\), where
\(v_0 = 15.3\) \(m/s\) \& \(a = -3\) \(m/s^2\)

The last part is for the situation in which the vehicle has come to a
complete stop (\(v = 0\)):

\(x = x_{final}\)

Now that we understand what equations we need to use, we can create
definitions for functions that will plot our equations of motion (EoMs)
for any given constants. Since most of the parameters are constant, the
definitions will only require the initial position and total time
interval.

    \begin{Verbatim}[commandchars=\\\{\}]
{\color{incolor}In [{\color{incolor}16}]:} \PY{k}{def} \PY{n+nf}{eqdrive}\PY{p}{(}\PY{n}{xinitial}\PY{p}{,} \PY{n}{timedrive}\PY{p}{)}\PY{p}{:}
             \PY{n}{t} \PY{o}{=} \PY{n}{np}\PY{o}{.}\PY{n}{arange}\PY{p}{(}\PY{l+m+mi}{0} \PY{p}{,} \PY{n}{timedrive} \PY{o}{+} \PY{l+m+mf}{0.1} \PY{p}{,} \PY{l+m+mf}{0.1}\PY{p}{)}
             \PY{n}{x} \PY{o}{=} \PY{p}{[}\PY{p}{]}
             \PY{n}{v} \PY{o}{=} \PY{p}{[}\PY{p}{]}
             \PY{k}{for} \PY{n}{i} \PY{o+ow}{in} \PY{n}{t}\PY{p}{:}
                 \PY{n}{i} \PY{o}{=} \PY{n}{xinitial} \PY{o}{+} \PY{n}{v0}\PY{o}{*}\PY{n}{i}
                 \PY{n}{x}\PY{o}{.}\PY{n}{append}\PY{p}{(}\PY{n}{i}\PY{p}{)}
                 \PY{n}{v}\PY{o}{.}\PY{n}{append}\PY{p}{(}\PY{n}{v0}\PY{p}{)}
             \PY{k}{return} \PY{n}{x}\PY{p}{,} \PY{n}{v}
         
         \PY{k}{def} \PY{n+nf}{eqbrake}\PY{p}{(}\PY{n}{xinitialbrake}\PY{p}{,} \PY{n}{timebrake}\PY{p}{)}\PY{p}{:}
             \PY{n}{reactioninterval} \PY{o}{=} \PY{n}{np}\PY{o}{.}\PY{n}{arange}\PY{p}{(}\PY{l+m+mi}{0} \PY{p}{,} \PY{n}{d} \PY{p}{,} \PY{l+m+mf}{0.1}\PY{p}{)}
             \PY{n}{postreaction} \PY{o}{=} \PY{n}{np}\PY{o}{.}\PY{n}{arange}\PY{p}{(}\PY{n}{d} \PY{p}{,} \PY{n}{timebrake} \PY{o}{+} \PY{l+m+mf}{0.1}\PY{p}{,} \PY{l+m+mf}{0.1}\PY{p}{)}
             \PY{n}{position} \PY{o}{=} \PY{p}{[}\PY{p}{]}
             \PY{n}{velocity} \PY{o}{=} \PY{p}{[}\PY{p}{]}
             \PY{n}{tzerov} \PY{o}{=} \PY{p}{(}\PY{o}{\PYZhy{}} \PY{n}{v0} \PY{o}{/} \PY{n}{a}\PY{p}{)} \PY{o}{+} \PY{n}{d}
             \PY{n}{finalprereactionx} \PY{o}{=} \PY{n}{xinitialbrake} \PY{o}{+} \PY{n}{v0}\PY{o}{*}\PY{n}{d}
             \PY{k}{for} \PY{n}{ii} \PY{o+ow}{in} \PY{n}{reactioninterval}\PY{p}{:}
                 \PY{n}{prereactionx} \PY{o}{=} \PY{n}{xinitialbrake} \PY{o}{+} \PY{n}{v0}\PY{o}{*}\PY{n}{ii}
                 \PY{n}{prereactionv} \PY{o}{=} \PY{n}{v0}
                 \PY{n}{position}\PY{o}{.}\PY{n}{append}\PY{p}{(}\PY{n}{prereactionx}\PY{p}{)}
                 \PY{n}{velocity}\PY{o}{.}\PY{n}{append}\PY{p}{(}\PY{n}{prereactionv}\PY{p}{)}
             \PY{k}{for} \PY{n}{i0} \PY{o+ow}{in} \PY{n}{postreaction}\PY{p}{:}
                 \PY{k}{if} \PY{n}{i0} \PY{o}{\PYZlt{}} \PY{n}{tzerov}\PY{p}{:}
                     \PY{n}{postreactionv} \PY{o}{=} \PY{n}{v0} \PY{o}{+} \PY{n}{a} \PY{o}{*} \PY{p}{(}\PY{n}{i0} \PY{o}{\PYZhy{}} \PY{n}{d}\PY{p}{)}
                     \PY{n}{postreactionx} \PY{o}{=} \PY{n}{finalprereactionx} \PY{o}{+} \PY{p}{(}\PY{n}{v0} \PY{o}{*} \PY{p}{(}\PY{n}{i0} \PY{o}{\PYZhy{}} \PY{n}{d}\PY{p}{)}\PY{p}{)} \PY{o}{+} \PY{p}{(}\PY{p}{(}\PY{n}{a}\PY{o}{/}\PY{l+m+mi}{2}\PY{p}{)} \PY{o}{*} \PY{p}{(}\PY{n}{i0} \PY{o}{\PYZhy{}} \PY{n}{d}\PY{p}{)}\PY{o}{*}\PY{o}{*}\PY{l+m+mi}{2}\PY{p}{)}
                     \PY{n}{position}\PY{o}{.}\PY{n}{append}\PY{p}{(}\PY{n}{postreactionx}\PY{p}{)}
                     \PY{n}{velocity}\PY{o}{.}\PY{n}{append}\PY{p}{(}\PY{n}{postreactionv}\PY{p}{)}
                 \PY{k}{if} \PY{n}{i0} \PY{o}{\PYZgt{}}\PY{o}{=} \PY{n}{tzerov}\PY{p}{:}
                     \PY{n}{xi} \PY{o}{=} \PY{n}{finalprereactionx} \PY{o}{+} \PY{n}{v0} \PY{o}{*} \PY{p}{(}\PY{n}{tzerov} \PY{o}{\PYZhy{}} \PY{n}{d}\PY{p}{)} \PY{o}{+} \PY{p}{(}\PY{n}{a}\PY{o}{/}\PY{l+m+mi}{2}\PY{p}{)} \PY{o}{*} \PY{p}{(}\PY{n}{tzerov} \PY{o}{\PYZhy{}} \PY{n}{d}\PY{p}{)}\PY{o}{*}\PY{o}{*}\PY{l+m+mi}{2}
                     \PY{n}{vi} \PY{o}{=} \PY{l+m+mi}{0}
                     \PY{n}{position}\PY{o}{.}\PY{n}{append}\PY{p}{(}\PY{n}{xi}\PY{p}{)}
                     \PY{n}{velocity}\PY{o}{.}\PY{n}{append}\PY{p}{(}\PY{n}{vi}\PY{p}{)}
             \PY{k}{return} \PY{n}{position}\PY{p}{,} \PY{n}{velocity}
\end{Verbatim}

    \subsection{2. Plotting/Analyzing EoMs for Different Initial
Positions}\label{plottinganalyzing-eoms-for-different-initial-positions}

    \subsubsection{Part A}\label{part-a}

    We will begin by creating two definitions, \(plotposition\) and
\(plotspeed\), which will plot the equations of motion for us. We will
add a red dashed line on both plots to indicate when the three second
interval has passed. We will also add a gray block to indicate when the
vehicle is in the interseection. Lastly, we will create a definition
that calls both previous definitions and plots them side by side.

    \begin{Verbatim}[commandchars=\\\{\}]
{\color{incolor}In [{\color{incolor}19}]:} \PY{k}{def} \PY{n+nf}{plotposition}\PY{p}{(}\PY{n}{taxis}\PY{p}{,} \PY{n}{initialposition}\PY{p}{)}\PY{p}{:}
             \PY{n}{totaltime} \PY{o}{=} \PY{n}{np}\PY{o}{.}\PY{n}{arange}\PY{p}{(}\PY{l+m+mi}{0}\PY{p}{,} \PY{n}{taxis} \PY{o}{+} \PY{l+m+mf}{0.1}\PY{p}{,} \PY{l+m+mf}{0.1}\PY{p}{)}
             \PY{n}{plt}\PY{o}{.}\PY{n}{rc}\PY{p}{(}\PY{l+s+s1}{\PYZsq{}}\PY{l+s+s1}{grid}\PY{l+s+s1}{\PYZsq{}}\PY{p}{,} \PY{n}{linestyle}\PY{o}{=}\PY{l+s+s2}{\PYZdq{}}\PY{l+s+s2}{dotted}\PY{l+s+s2}{\PYZdq{}}\PY{p}{,} \PY{n}{color}\PY{o}{=}\PY{l+s+s1}{\PYZsq{}}\PY{l+s+s1}{black}\PY{l+s+s1}{\PYZsq{}}\PY{p}{,} \PY{n}{alpha}\PY{o}{=}\PY{l+m+mf}{0.2}\PY{p}{)}
             \PY{n}{plt}\PY{o}{.}\PY{n}{grid}\PY{p}{(}\PY{k+kc}{True}\PY{p}{)}
             \PY{n}{plt}\PY{o}{.}\PY{n}{plot}\PY{p}{(}\PY{n}{totaltime}\PY{p}{,} \PY{n}{eqbrake}\PY{p}{(}\PY{n}{initialposition}\PY{p}{,} \PY{n}{taxis}\PY{p}{)}\PY{p}{[}\PY{l+m+mi}{0}\PY{p}{]}\PY{p}{,} \PY{l+s+s1}{\PYZsq{}}\PY{l+s+s1}{\PYZhy{}}\PY{l+s+s1}{\PYZsq{}}\PY{p}{,} \PY{n}{label} \PY{o}{=} \PY{l+s+s1}{\PYZsq{}}\PY{l+s+s1}{Brake}\PY{l+s+s1}{\PYZsq{}}\PY{p}{)}
             \PY{n}{plt}\PY{o}{.}\PY{n}{plot}\PY{p}{(}\PY{n}{totaltime}\PY{p}{,} \PY{n}{eqdrive}\PY{p}{(}\PY{n}{initialposition}\PY{p}{,} \PY{n}{taxis}\PY{p}{)}\PY{p}{[}\PY{l+m+mi}{0}\PY{p}{]}\PY{p}{,} \PY{l+s+s1}{\PYZsq{}}\PY{l+s+s1}{\PYZhy{}}\PY{l+s+s1}{\PYZsq{}}\PY{p}{,} \PY{n}{label} \PY{o}{=} \PY{l+s+s1}{\PYZsq{}}\PY{l+s+s1}{Drive}\PY{l+s+s1}{\PYZsq{}}\PY{p}{)}
             \PY{n}{plt}\PY{o}{.}\PY{n}{plot}\PY{p}{(}\PY{p}{[}\PY{n}{tau}\PY{p}{,} \PY{n}{tau}\PY{p}{]}\PY{p}{,} \PY{p}{[}\PY{n}{initialposition}\PY{p}{,} \PY{l+m+mf}{1.05}\PY{o}{*}\PY{n}{w}\PY{p}{]}\PY{p}{,} \PY{l+s+s2}{\PYZdq{}}\PY{l+s+s2}{\PYZhy{}\PYZhy{}}\PY{l+s+s2}{\PYZdq{}}\PY{p}{,} \PY{n}{color}\PY{o}{=}\PY{l+s+s2}{\PYZdq{}}\PY{l+s+s2}{red}\PY{l+s+s2}{\PYZdq{}}\PY{p}{,} \PY{n}{lw}\PY{o}{=}\PY{l+m+mi}{1}\PY{p}{)}
             \PY{n}{plt}\PY{o}{.}\PY{n}{fill\PYZus{}between}\PY{p}{(}\PY{p}{[}\PY{l+m+mi}{0}\PY{p}{,} \PY{l+m+mi}{7}\PY{p}{]}\PY{p}{,} \PY{p}{[}\PY{n}{w}\PY{p}{,} \PY{n}{w}\PY{p}{]}\PY{p}{,} \PY{n}{color}\PY{o}{=}\PY{l+s+s2}{\PYZdq{}}\PY{l+s+s2}{black}\PY{l+s+s2}{\PYZdq{}}\PY{p}{,} \PY{n}{alpha}\PY{o}{=}\PY{l+m+mf}{0.3}\PY{p}{)}
             \PY{n}{plt}\PY{o}{.}\PY{n}{xlabel}\PY{p}{(}\PY{l+s+s1}{\PYZsq{}}\PY{l+s+s1}{Time t (s)}\PY{l+s+s1}{\PYZsq{}}\PY{p}{)}
             \PY{n}{plt}\PY{o}{.}\PY{n}{ylabel}\PY{p}{(}\PY{l+s+s1}{\PYZsq{}}\PY{l+s+s1}{Position x(t) (m)}\PY{l+s+s1}{\PYZsq{}}\PY{p}{)}
             \PY{n}{plt}\PY{o}{.}\PY{n}{title}\PY{p}{(}\PY{l+s+s1}{\PYZsq{}}\PY{l+s+s1}{\PYZdl{}x(t) = x\PYZus{}0 + v\PYZus{}0t \PYZhy{} (a/2) t\PYZca{}2\PYZdl{}}\PY{l+s+s1}{\PYZsq{}}\PY{p}{)}
             \PY{n}{plt}\PY{o}{.}\PY{n}{legend}\PY{p}{(}\PY{p}{)}
             
         \PY{k}{def} \PY{n+nf}{plotspeed}\PY{p}{(}\PY{n}{xaxis}\PY{p}{,} \PY{n}{initialpos}\PY{p}{)}\PY{p}{:}
             \PY{n}{totalt} \PY{o}{=} \PY{n}{np}\PY{o}{.}\PY{n}{arange}\PY{p}{(}\PY{l+m+mi}{0}\PY{p}{,} \PY{n}{xaxis} \PY{o}{+} \PY{l+m+mf}{0.1}\PY{p}{,} \PY{l+m+mf}{0.1}\PY{p}{)}
             \PY{n}{plt}\PY{o}{.}\PY{n}{rc}\PY{p}{(}\PY{l+s+s1}{\PYZsq{}}\PY{l+s+s1}{grid}\PY{l+s+s1}{\PYZsq{}}\PY{p}{,} \PY{n}{linestyle}\PY{o}{=}\PY{l+s+s2}{\PYZdq{}}\PY{l+s+s2}{dotted}\PY{l+s+s2}{\PYZdq{}}\PY{p}{,} \PY{n}{color}\PY{o}{=}\PY{l+s+s1}{\PYZsq{}}\PY{l+s+s1}{black}\PY{l+s+s1}{\PYZsq{}}\PY{p}{,} \PY{n}{alpha}\PY{o}{=}\PY{l+m+mf}{0.2}\PY{p}{)}
             \PY{n}{plt}\PY{o}{.}\PY{n}{grid}\PY{p}{(}\PY{k+kc}{True}\PY{p}{)}
             \PY{n}{plt}\PY{o}{.}\PY{n}{plot}\PY{p}{(}\PY{n}{totalt}\PY{p}{,} \PY{n}{eqbrake}\PY{p}{(}\PY{n}{initialpos}\PY{p}{,} \PY{n}{xaxis}\PY{p}{)}\PY{p}{[}\PY{l+m+mi}{1}\PY{p}{]}\PY{p}{,} \PY{l+s+s1}{\PYZsq{}}\PY{l+s+s1}{\PYZhy{}}\PY{l+s+s1}{\PYZsq{}}\PY{p}{,} \PY{n}{label} \PY{o}{=} \PY{l+s+s1}{\PYZsq{}}\PY{l+s+s1}{Brake}\PY{l+s+s1}{\PYZsq{}}\PY{p}{)}
             \PY{n}{plt}\PY{o}{.}\PY{n}{plot}\PY{p}{(}\PY{n}{totalt}\PY{p}{,} \PY{n}{eqdrive}\PY{p}{(}\PY{n}{initialpos}\PY{p}{,} \PY{n}{xaxis}\PY{p}{)}\PY{p}{[}\PY{l+m+mi}{1}\PY{p}{]}\PY{p}{,} \PY{l+s+s1}{\PYZsq{}}\PY{l+s+s1}{\PYZhy{}}\PY{l+s+s1}{\PYZsq{}}\PY{p}{,} \PY{n}{label} \PY{o}{=} \PY{l+s+s1}{\PYZsq{}}\PY{l+s+s1}{Drive}\PY{l+s+s1}{\PYZsq{}}\PY{p}{)}
             \PY{n}{plt}\PY{o}{.}\PY{n}{plot}\PY{p}{(}\PY{p}{[}\PY{n}{tau}\PY{p}{,} \PY{n}{tau}\PY{p}{]}\PY{p}{,} \PY{p}{[}\PY{n}{initialpos}\PY{o}{/}\PY{l+m+mi}{200}\PY{p}{,} \PY{n}{w}\PY{o}{/}\PY{l+m+mf}{1.8}\PY{p}{]}\PY{p}{,} \PY{l+s+s2}{\PYZdq{}}\PY{l+s+s2}{\PYZhy{}\PYZhy{}}\PY{l+s+s2}{\PYZdq{}}\PY{p}{,} \PY{n}{color}\PY{o}{=}\PY{l+s+s2}{\PYZdq{}}\PY{l+s+s2}{red}\PY{l+s+s2}{\PYZdq{}}\PY{p}{,} \PY{n}{lw}\PY{o}{=}\PY{l+m+mi}{1}\PY{p}{)}
             \PY{n}{plt}\PY{o}{.}\PY{n}{xlabel}\PY{p}{(}\PY{l+s+s1}{\PYZsq{}}\PY{l+s+s1}{Time t (s)}\PY{l+s+s1}{\PYZsq{}}\PY{p}{)}
             \PY{n}{plt}\PY{o}{.}\PY{n}{ylabel}\PY{p}{(}\PY{l+s+s1}{\PYZsq{}}\PY{l+s+s1}{Velocity(t) (m/s)}\PY{l+s+s1}{\PYZsq{}}\PY{p}{)}
             \PY{n}{plt}\PY{o}{.}\PY{n}{title}\PY{p}{(}\PY{l+s+s1}{\PYZsq{}}\PY{l+s+s1}{\PYZdl{}v(t) = v\PYZus{}0 + at\PYZdl{}}\PY{l+s+s1}{\PYZsq{}}\PY{p}{)}
             \PY{n}{plt}\PY{o}{.}\PY{n}{legend}\PY{p}{(}\PY{p}{)}
             
         \PY{k}{def} \PY{n+nf}{ploteoms}\PY{p}{(}\PY{n}{x0}\PY{p}{,} \PY{n}{tt}\PY{p}{)}\PY{p}{:}
             \PY{n}{plt}\PY{o}{.}\PY{n}{subplot}\PY{p}{(}\PY{l+m+mi}{1}\PY{p}{,} \PY{l+m+mi}{2}\PY{p}{,} \PY{l+m+mi}{1}\PY{p}{)}
             \PY{n}{plotposition}\PY{p}{(}\PY{n}{tt}\PY{p}{,} \PY{n}{x0}\PY{p}{)}
             \PY{n}{plt}\PY{o}{.}\PY{n}{subplot}\PY{p}{(}\PY{l+m+mi}{1}\PY{p}{,} \PY{l+m+mi}{2}\PY{p}{,} \PY{l+m+mi}{2}\PY{p}{)}
             \PY{n}{plotspeed}\PY{p}{(}\PY{n}{tt}\PY{p}{,} \PY{n}{x0}\PY{p}{)}
             \PY{n}{plt}\PY{o}{.}\PY{n}{tight\PYZus{}layout}\PY{p}{(}\PY{p}{)}
             \PY{n}{plt}\PY{o}{.}\PY{n}{show}\PY{p}{(}\PY{p}{)}
\end{Verbatim}

    \subsubsection{Part B}\label{part-b}

    Now that we can plot the EoMs, the first scenario we will consider is
that where the traffic signal changes to yellow while the driver is 30
meters from the intersection.

    \begin{Verbatim}[commandchars=\\\{\}]
{\color{incolor}In [{\color{incolor}20}]:} \PY{n}{ploteoms}\PY{p}{(}\PY{o}{\PYZhy{}}\PY{l+m+mi}{30}\PY{p}{,} \PY{l+m+mi}{7}\PY{p}{)}
\end{Verbatim}

    \begin{center}
    \adjustimage{max size={0.9\linewidth}{0.9\paperheight}}{output_13_0.png}
    \end{center}
    { \hspace*{\fill} \\}
    
    \(Figure\) 1. EoMs plots for the first scenario (\(x_0 = -30\)).

    From figure 1 we can see that the driver approaches a dangerous
situation in both cases. The next scenario has the light change color at
70 meters from the intersection.

    \begin{Verbatim}[commandchars=\\\{\}]
{\color{incolor}In [{\color{incolor}21}]:} \PY{n}{ploteoms}\PY{p}{(}\PY{o}{\PYZhy{}}\PY{l+m+mi}{70}\PY{p}{,} \PY{l+m+mi}{7}\PY{p}{)}
\end{Verbatim}

    \begin{center}
    \adjustimage{max size={0.9\linewidth}{0.9\paperheight}}{output_16_0.png}
    \end{center}
    { \hspace*{\fill} \\}
    
    \(Figure\) 2. EoMs plots for the second scenario (\(x_0=-70\)).

    From figure 2 we can see that in both cases, the three seconds pass
before the driver gets to the intersection. \(t = 3\). Thus, both cases
seem rather safe. The last case we will consider for now is the case
where the light changes at \(x_0 = -0.5\)m.

    \begin{Verbatim}[commandchars=\\\{\}]
{\color{incolor}In [{\color{incolor}22}]:} \PY{n}{ploteoms}\PY{p}{(}\PY{o}{\PYZhy{}}\PY{l+m+mf}{0.5}\PY{p}{,}\PY{l+m+mi}{7}\PY{p}{)}
\end{Verbatim}

    \begin{center}
    \adjustimage{max size={0.9\linewidth}{0.9\paperheight}}{output_19_0.png}
    \end{center}
    { \hspace*{\fill} \\}
    
    \(Figure\) 3. EoMs plots for the third case (\(x_0=-0.5\)).

    From figure 3 we see that in either case, the situation at hand seems
safe.

    \subsubsection{Part C}\label{part-c}

    We can establish a set of rules that will help us identify which
case/decision is safe based on the vehicle's trajectory. We will
classify each outcome as safe if,

for driving: \(x>W\) for \(t>\tau\), or

for braking: \(x<0\) for all \(t\).

All other cases will be classified as dangerous. The outcome for a given
\(x_0\) will be safe overall if there exists at least one safe decision.

    Analyzing figure 1, we see that neither of the safe conditions are met
for either decision. Thus, the outcome for \(x_0=-30m\) is dangerous.
Figure 2 shows a distinct outcome. The vehicle does not clear the
intersection before the three seconds, but it does come to a full stop
before the intersection if it were to brake. Thus, the outcome for
\(x_0=-70m\) is safe. Lastly, figure 3 is also different. The vehicle
fails to stop before the intersection if it decides to brake, but can
clear it if it keeps moving at constant velocity. Thus, the outcome for
\(x_0=-0.5m\) is safe.

    \subsubsection{Part D}\label{part-d}

    We can map every possible outcome by considering three zones. The first
zone is for \(x > W\) for \(t>\tau\) (the \(x_0^A\) zone). The second
zone is for \(x<0\) for all \(t\) (the \(x_0^B\) zone). The third zone
is for every other possible value of \(x_0\). Thus, zones one and two
are for safe values of \(x_0\) while the third zone isn't.

    \begin{Verbatim}[commandchars=\\\{\}]
{\color{incolor}In [{\color{incolor}43}]:} \PY{n}{x0s} \PY{o}{=} \PY{n}{np}\PY{o}{.}\PY{n}{arange}\PY{p}{(}\PY{o}{\PYZhy{}}\PY{l+m+mi}{100}\PY{p}{,} \PY{l+m+mf}{0.5}\PY{p}{,} \PY{l+m+mf}{0.5}\PY{p}{)}
         \PY{n}{classifiedx0s} \PY{o}{=} \PY{p}{[}\PY{p}{]}
         \PY{n}{finalxs} \PY{o}{=} \PY{p}{[}\PY{p}{]} 
         \PY{n}{dilemma} \PY{o}{=} \PY{p}{[}\PY{p}{]}  
         \PY{k}{for} \PY{n}{i} \PY{o+ow}{in} \PY{n}{x0s}\PY{p}{:}
             \PY{n}{finalstop} \PY{o}{=} \PY{n}{i} \PY{o}{+} \PY{p}{(}\PY{n}{v0} \PY{o}{*} \PY{n}{d}\PY{p}{)} \PY{o}{+} \PY{p}{(}\PY{n}{v0} \PY{o}{*} \PY{p}{(}\PY{n}{tau} \PY{o}{\PYZhy{}} \PY{n}{d}\PY{p}{)}\PY{p}{)} \PY{o}{+} \PY{p}{(}\PY{p}{(}\PY{n}{a}\PY{o}{/}\PY{l+m+mi}{2}\PY{p}{)} \PY{o}{*} \PY{p}{(}\PY{n}{tau} \PY{o}{\PYZhy{}} \PY{n}{d}\PY{p}{)}\PY{o}{*}\PY{o}{*}\PY{l+m+mi}{2}\PY{p}{)}
             \PY{n}{finalx} \PY{o}{=} \PY{n}{i} \PY{o}{+} \PY{n}{v0}\PY{o}{*}\PY{n}{tau}
             \PY{k}{if} \PY{n}{finalstop} \PY{o}{\PYZlt{}} \PY{l+m+mi}{0}\PY{p}{:}
                 \PY{n}{finalxs}\PY{o}{.}\PY{n}{append}\PY{p}{(}\PY{n}{finalstop}\PY{p}{)}
                 \PY{n}{classifiedx0s}\PY{o}{.}\PY{n}{append}\PY{p}{(}\PY{l+s+s1}{\PYZsq{}}\PY{l+s+s1}{green}\PY{l+s+s1}{\PYZsq{}}\PY{p}{)}
             \PY{k}{if} \PY{n}{finalx} \PY{o}{\PYZgt{}} \PY{n}{w}\PY{p}{:}
                 \PY{n}{classifiedx0s}\PY{o}{.}\PY{n}{append}\PY{p}{(}\PY{l+s+s1}{\PYZsq{}}\PY{l+s+s1}{blue}\PY{l+s+s1}{\PYZsq{}}\PY{p}{)}
                 \PY{n}{finalxs}\PY{o}{.}\PY{n}{append}\PY{p}{(}\PY{n}{finalx}\PY{p}{)}
             \PY{k}{if} \PY{n}{finalx} \PY{o}{\PYZlt{}} \PY{n}{w} \PY{o+ow}{and} \PY{n}{finalstop} \PY{o}{\PYZgt{}} \PY{l+m+mi}{0}\PY{p}{:}
                 \PY{n}{classifiedx0s}\PY{o}{.}\PY{n}{append}\PY{p}{(}\PY{l+s+s1}{\PYZsq{}}\PY{l+s+s1}{red}\PY{l+s+s1}{\PYZsq{}}\PY{p}{)}
                 \PY{n}{finalxs}\PY{o}{.}\PY{n}{append}\PY{p}{(}\PY{l+m+mi}{15}\PY{p}{)}
                 \PY{n}{dilemma}\PY{o}{.}\PY{n}{append}\PY{p}{(}\PY{n}{i}\PY{p}{)}
         \PY{n}{plt}\PY{o}{.}\PY{n}{scatter}\PY{p}{(}\PY{n}{x0s}\PY{p}{,} \PY{n}{finalxs}\PY{p}{,} \PY{n}{c} \PY{o}{=} \PY{n}{classifiedx0s}\PY{p}{)}
         \PY{n}{plt}\PY{o}{.}\PY{n}{rc}\PY{p}{(}\PY{l+s+s1}{\PYZsq{}}\PY{l+s+s1}{grid}\PY{l+s+s1}{\PYZsq{}}\PY{p}{,} \PY{n}{linestyle}\PY{o}{=}\PY{l+s+s2}{\PYZdq{}}\PY{l+s+s2}{dotted}\PY{l+s+s2}{\PYZdq{}}\PY{p}{,} \PY{n}{color}\PY{o}{=}\PY{l+s+s1}{\PYZsq{}}\PY{l+s+s1}{black}\PY{l+s+s1}{\PYZsq{}}\PY{p}{,} \PY{n}{alpha}\PY{o}{=}\PY{l+m+mf}{0.2}\PY{p}{)}
         \PY{n}{plt}\PY{o}{.}\PY{n}{grid}\PY{p}{(}\PY{k+kc}{True}\PY{p}{)}
         \PY{n}{plt}\PY{o}{.}\PY{n}{fill\PYZus{}between}\PY{p}{(}\PY{p}{[}\PY{n+nb}{min}\PY{p}{(}\PY{n}{dilemma}\PY{p}{)}\PY{p}{,} \PY{n+nb}{max}\PY{p}{(}\PY{n}{dilemma}\PY{p}{)}\PY{p}{]}\PY{p}{,} \PY{p}{[}\PY{n}{w}\PY{p}{,} \PY{n}{w}\PY{p}{]}\PY{p}{,} \PY{n}{color}\PY{o}{=}\PY{l+s+s2}{\PYZdq{}}\PY{l+s+s2}{black}\PY{l+s+s2}{\PYZdq{}}\PY{p}{,} \PY{n}{alpha}\PY{o}{=}\PY{l+m+mf}{0.3}\PY{p}{,} \PY{n}{label} \PY{o}{=} \PY{l+s+s1}{\PYZsq{}}\PY{l+s+s1}{Dilemma Zone (S)}\PY{l+s+s1}{\PYZsq{}}\PY{p}{)}
         \PY{n}{plt}\PY{o}{.}\PY{n}{legend}\PY{p}{(}\PY{p}{)}
         \PY{n}{plt}\PY{o}{.}\PY{n}{xlabel}\PY{p}{(}\PY{l+s+s1}{\PYZsq{}}\PY{l+s+s1}{Values of \PYZdl{}x\PYZus{}0\PYZdl{} (m)}\PY{l+s+s1}{\PYZsq{}}\PY{p}{)}
         \PY{n}{plt}\PY{o}{.}\PY{n}{ylabel}\PY{p}{(}\PY{l+s+s1}{\PYZsq{}}\PY{l+s+s1}{Final position (m)}\PY{l+s+s1}{\PYZsq{}}\PY{p}{)}
         \PY{n}{plt}\PY{o}{.}\PY{n}{title}\PY{p}{(}\PY{l+s+s1}{\PYZsq{}}\PY{l+s+s1}{Classification of \PYZdl{}x\PYZus{}0\PYZdl{} values}\PY{l+s+s1}{\PYZsq{}}\PY{p}{)}
         \PY{n}{ax} \PY{o}{=} \PY{n}{plt}\PY{o}{.}\PY{n}{gca}\PY{p}{(}\PY{p}{)}
         \PY{n}{plt}\PY{o}{.}\PY{n}{text}\PY{p}{(}\PY{l+m+mi}{1}\PY{p}{,} \PY{l+m+mi}{0}\PY{p}{,} \PY{l+s+s1}{\PYZsq{}}\PY{l+s+s1}{S = 22.5 m}\PY{l+s+s1}{\PYZsq{}}\PY{p}{,} \PY{n}{horizontalalignment}\PY{o}{=}\PY{l+s+s1}{\PYZsq{}}\PY{l+s+s1}{right}\PY{l+s+s1}{\PYZsq{}}\PY{p}{,}
              \PY{n}{verticalalignment}\PY{o}{=}\PY{l+s+s1}{\PYZsq{}}\PY{l+s+s1}{bottom}\PY{l+s+s1}{\PYZsq{}}\PY{p}{,}
              \PY{n}{transform} \PY{o}{=} \PY{n}{ax}\PY{o}{.}\PY{n}{transAxes}\PY{p}{)}
         \PY{n+nb}{print}\PY{p}{(}\PY{l+s+s1}{\PYZsq{}}\PY{l+s+s1}{Dilemma zone goes from}\PY{l+s+s1}{\PYZsq{}}\PY{p}{,} \PY{n+nb}{min}\PY{p}{(}\PY{n}{dilemma}\PY{p}{)}\PY{p}{,} \PY{l+s+s1}{\PYZsq{}}\PY{l+s+s1}{m to}\PY{l+s+s1}{\PYZsq{}}\PY{p}{,} \PY{n+nb}{max}\PY{p}{(}\PY{n}{dilemma}\PY{p}{)}\PY{p}{,}\PY{l+s+s1}{\PYZsq{}}\PY{l+s+s1}{m}\PY{l+s+s1}{\PYZsq{}}\PY{p}{)}
\end{Verbatim}

    \begin{Verbatim}[commandchars=\\\{\}]
Dilemma zone goes from -38.5 m to -16.0 m

    \end{Verbatim}

    \begin{center}
    \adjustimage{max size={0.9\linewidth}{0.9\paperheight}}{output_27_1.png}
    \end{center}
    { \hspace*{\fill} \\}
    
    \(Figure\) 4. Classification of \(x_0\) values where green corresponds
to the \(x_0^B\) region (safe to brake), blue to the \(x_0^A\) (safe to
drive), and gray box with red line to all values that lie in the dilemma
zone (\(S = x_0^A - x_0^B\)).

    We can see from the plot that the dilemma zone ranges from -38.5m to
-16.0m. Thus, every value within these two points would mean that the
vehicle is approaching a dangerous scenario regardless of the decision
the driver makes.

    \subsubsection{Part E}\label{part-e}

    To solve the problem analytically, we begin by calculating the smallest
value of \(x_0\) that falls in the \(x_0^A\) zone. All we have to do is
rearrange the EoM for \(x\) due to constant velocity, \(15.3\) \(m/s\),

\(x = x_0^A + vt\).

We use \(x=W\) since that is the point where the inersection ends and
\(t=3s\) since that's how much time we are allowed to clear the
intersection.

\(x_0^A = W - v_0*\tau = -16m\)

    The next step is to determine the largest possible value of \(x_0^B\).
This time, we split the EoM into two parts, the constant speed and the
deceleration. The deceleration is as follows,

\(x=x_0^{B1} + v_0t + \frac{a}{2} t^2\)

    We let \(x=0\) and \(t=\tau - \delta\) and we solve for \(x_0^{B1}\).

    \(x_0^{B1} = -v_0 (\tau - \delta) - \frac{a}{2} (\tau - \delta)^2 = -26.2m\)

    To account for \(\delta\), we find the distance traveled at a constant
velocity,

    \(x_0^{B2} = v_0 \delta = 12.2m\)

    \(x_0^B = x_0^{B1} - x_0^{B2} = -38.5m\)

    Finally, we find the dilemma zone, S,

\(S = x_0^A - x_0^B = 22.5m\)

    \subsubsection{Part F}\label{part-f}

    The dilemma zone is shifted by a change in \(v_0\), but the range
remains the same. Thus, \(S\) is independant of \(v_0\).

    \begin{Verbatim}[commandchars=\\\{\}]
{\color{incolor}In [{\color{incolor} }]:} 
\end{Verbatim}


    % Add a bibliography block to the postdoc
    
    
    
    \end{document}
